\documentclass[11pt,a4paper,sans]{moderncv}

\moderncvstyle{fancy}
\moderncvcolor{blue}

\usepackage{amsmath,amsthm,amssymb}
\usepackage{mathtext}
\usepackage[T1,T2A]{fontenc}
\usepackage[utf8]{inputenc}
\usepackage[english,russian]{babel}
\usepackage[none]{hyphenat}
\usepackage[scale=0.75]{geometry}

\newcommand{\textsharp}{$\sharp$}
\def\cpp{C/C{}\texttt{++}}

\phone[mobile]{по запросу}
\name{Собко}{Сергей Сергеевич}
\title{Ведущий системный архитектор}

\extrainfo{* 24 ноября 1989}

\social[linkedin]{bug2bug}
\social[github]{profitware}

\photo[64pt][0.4pt]{../img/picture.jpg}

%\quote{Hello, world!}


\begin{document}
\makecvtitle

\section{Образование}
\cventry{2010--2016}{Специалист, 230101 Вычислительные машины, комплексы,
системы и сети}{Московский Институт Электроники и Математики НИУ ВШЭ}{Москва}{}{}


\section{Языки}
\cvlistitem{Английский (Upper Intermediate)}
\cvlistitem{Русский (Родной)}


\section{Техническая экспертиза}
\subsection{Профессионально}
\cvline{языки программирования}{Python, JavaScript, Clojure, Matlab, Delphi}
\cvline{технологии}{Django, Flask, Tornado, MySQL, SQLite, Emacs,
\LaTeX{}, Bash Scripting, Git, Subversion, PostgreSQL, MongoDB, jQuery,
Bootstrap, Windows, Debian \& Red Hat Linux, Mac OS X, JSON, Ant, XSLT,
Regexr, HTML, CSS, Docker, Kubernetes, Helm, Apache Camel, Red Hat OpenShift}
\subsection{Есть опыт}
\cvline{языки программирования}{PHP, Java, C$^\sharp$, $\cpp$, Erlang, Groovy, Haskell,
Scala} \cvline{технологии}{Apache Kafka, Nginx, Flash, GIMP, SQL Server, Oracle,
Eclipse, .NET, Spring, Rancher, MicroK8s, OpenStack}


\section{Текущая позиция}
\cventry{2019--\textit{настоящее}}{Ведущий системный архитектор}{IBM}{Москва}{}{
Проектирование и реализация решений на базе открытого ПО:
\begin{itemize}
\item Red Hat Certified Architect in Infrastructure Level II (\url{https://www.redhat.com/rhtapps/services/verify/?certId=200-054-419})
\item Построение взаимоотношений с вендорами открытого ПО для программных продуктов Nginx, PostgreSQL, Apache Kafka и др.
\item Архитектура и проектирование решений на базе Red Hat OpenShift, Apache Kafka, PostgreSQL, решений по информационной безопасности, аудиту и мониторингу для компаний в сферах: банковской, государственной, телеком.
\item Пилотирование, тестирование, аудит и внедрение решений в различных организациях России, СНГ, Центральной и Восточной Европы.
\end{itemize}}

\section{Опыт работы}
\cventry{2016--2019}{Руководитель группы}{Positive Technologies}{Москва}{}{
Разработка систем управления Web Application Firewall (\url{https://af.ptsecurity.com/}) с использованием Python, Groovy и JavaScript.
\begin{itemize}
\item Проектирование и принятие решений по бэкэнд-части облачного PT AF.
\item Управление группой разработки PT AF с увеличением до 10 человек.
\item Разработка решения ``Автоматический тимлид'' для автоматизации управления командой и интеграционной бибилиотеки ``Flower'' в виде ПО с открытым исходным кодом (\url{https://github.com/PositiveTechnologies/flower}) -- более 100 звезд.
\item Разработка предметно-оринтированного языка (DSL) и интерпретатора для описания сетевой конфигурации и модели управления кластером.
\item Создания пакета устройства PT AF для интеграции с сетевой фабрикой Cisco ACI, презентация с представителями Cisco Systems на PHDays VI.
\item Реализация режима прозрачного проксирования для PT AF.
\end{itemize}
}

\cventry{2015--2016}{Старший программист}{Positive Technologies}{Москва}{}{
Разработка решения класса Web Application Firewall (\url{https://af.ptsecurity.com/})
с использованием Python (Flask, Twisted) и JavaScript.
\begin{itemize}
\item Реализация функционала авторизации и интеграции с Active Directory.
\item Реализация функциональности создания отчетов (атаки, предупреждения).
\item Взятие 3 места на конкурсе ``Digital Substation Takeover'' конференции Positive Hack Days V.
\end{itemize}}

\cventry{2013--2015}{Инженер-программист}{РБК}{Москва}{}{
Разработка решений по котировкам с использованием Python (Django, Tornado),
JavaScript, Postgres и Oracle.
\begin{itemize}
\item Quote.rbc.ru (\url{http://quote.rbc.ru/}) разработка и поддержка:
  \begin{itemize}
    \item доработка секции главных новостей (\url{http://quote.rbc.ru/topnews/});
    \item разработка секции сервисов и цен created - решение в формате интернет-магазина
    (\url{http://quote.rbc.ru/price/});
    \item реализация авторизации, регистрации и личного кабинета
    (\url{http://quote.rbc.ru/user/}) и интеграция с базой данных Oracle.
  \end{itemize}
\item Рынок наличной валюты (\url{http://cash.rbc.ru/}) - доработка, поддержка,
миграция с Oracle на Postgres.
\item Мониторинг котировок для Газпромбанка (\url{http://v2.gpbmon.rbc.ru/}).
\end{itemize}}

\cventry{2011--2013}{Программист}{BSS}{Москва}{}{
Разработка систем дистанционного банковского обслуживания (\url{http://www.bssys.com/}).
\begin{itemize}
\item Разработка специальных версий систем ДБО Частный Клиент для банков Nordea, Uniastrum, MTS, TFB с использованием Oracle, Delphi и jQuery.
\item Разработка функционала множественной подписи в ДБО ЧК.
\item Построение системы сборки с использованием Ant, Jython, Java и Javascript.
\end{itemize}}

\cventry{2010--2011}{Веб-разработчик}{DEKO Media}{Москва}{}{
Разработка веб-сайтов и внутренней ERP-системы с использованием Python, JavaScript и Oracle.
\begin{itemize}
\item Разработка ПО для рабочего места ресепшена и для управления складом.
\item Редизайн сайта DEKO Media (\url{http://deko-media.ru/}) и интеграция с системой управления складом.
\end{itemize}
}

\section{Разработка Open Source}
\cventry{2005--\textit{настоящее}}{Разработчик Open Source}{The Profitware Group}{}{}{
Участие в создании свободного и открытого программного обеспечения:
\begin{itemize}
\item Для Московского Института Электроники и Математики:
  \begin{itemize}
    \item CircuitryLib -- библиотка на Python для моделирования различных цифровых систем;
    \item SpiritLevel -- PoC для устройства на MicroPython;
    \item ErlangIO -- PoC для проверки возможности соединения Erlang VM с ядром GNU/Linux;
    \item mDSS -- Система принятия решений на Clojure;
    \item Cheque -- RESTful сервис для распознавания чеков на Flask.
  \end{itemize}
\item Productivity Desktop Environment для Fedora GNU/Linux.
\item Плагины на Leiningen для Red Hat OpenShift, для сбора статики из Ring и для управления версиями.
\item ProfitPlatform-NG -- распределенная RPC-система для Python и Erlang с использованием RabbitMQ.
\item Lisp Flavoured Erlang (LFE): реализация конечных ленивых последовательностей и имплементация cycle/1.
\item Небольшие улучшения в проектах SaltStack, Ferm, Gitlab Java API, HyLang, Rouge.
\end{itemize}
}


\section{Учебный и тренерский опыт}
\cventry{2016--\textit{настоящее}}{Приглашенный преподаватель}{Высшая Школа Экономики}{Москва}{}{
Курс по веб-девелопменту для магистерских программ Компьютерная лингвистика (\url{https://www.hse.ru/edu/courses/339567102}) и Цифровые методы в гуманитарных науках (\url{https://www.hse.ru/edu/courses/339567106}):
\begin{itemize}
\item верстка с использованием Bootstrap и других фреймворков;
\item создание веб-приложений с использованием Python (Flask, Celery);
\item разработка современных фронтэнд-приложений с помощью AngularJS;
\item упаковка веб-приложений в контейнеры Docker.
\end{itemize}}

\cventry{2016--\textit{настоящее}}{Тренер}{Django Girls}{Москва}{}{Тренировка участниц Django Girls Москва 2016 (\url{http://tceh.com/djangogirls/}).}

\section{Дипломная работа специалиста}
\cvline{Наименование}{Система управления конфигурацией Web Application Firewall кластера с гетерогенной сетевой структурой}
\cvline{Финальная оценка}{10 из 10}
\cvline{Аннотация}{\url{https://www.hse.ru/ru/edu/vkr/182863697}}
\cvline{Защита}{\url{https://www.youtube.com/watch?v=-pzmGtnNtz4}}

\section{Сертификации}
\cventry{2020}{RHCA in Infrastructure L2 (200-054-419)}{Red Hat}{Москва}{}{}
\cventry{2020}{RHCS in Camel Devel (200-054-419)}{Red Hat}{Москва}{}{}
\cventry{2020}{Red Hat Certified Engineer (200-054-419)}{Red Hat}{Москва}{}{}
\cventry{2020}{RHCSA (200-054-419)}{Red Hat}{Москва}{}{}
\cventry{2020}{Interskill - Mainframe Specialist - IBM Mainframe Environment - Fundamentals}{IBM}{}{}{}
\cventry{2020}{Architectural Thinking}{IBM}{}{}{}
\cventry{2020}{Beyond the Basics: Istio and IBM Cloud Kubernetes Service}{IBM}{}{}{}
\cventry{2020}{Containers, K8s and Istio on IBM Cloud}{IBM}{}{}{}
\cventry{2020}{IBM Recognized Speaker/ Presenter}{IBM}{}{}{}
\cventry{2020}{IBM Recognized Teacher/ Educator}{IBM}{}{}{}
\cventry{2020}{IBM Security Essentials for Architects}{IBM}{}{}{}
\cventry{2020}{Getting started with Microservices with Istio and IBM Cloud Kubernetes Service}{IBM}{}{}{}
\cventry{2020}{IBM Garage Method for Cloud Explorer}{IBM}{}{}{}
\cventry{2020}{Docker Essentials: A Developer Introduction}{IBM}{}{}{}
\cventry{2020}{GTS Architect Foundations}{IBM}{}{}{}
\cventry{2020}{IBM Cloud Kubernetes Service}{IBM}{}{}{}
\cventry{2020}{IBM Services Platform with Watson}{IBM}{}{}{}

\cventry{2020}{LFC210: Fundamentals of Professional Open Source Management}{The Linux Foundation}{}{}{}
\cventry{2020}{Security and Privacy by Design}{IBM}{}{}{}
\cventry{2020}{Think Like a Hacker}{IBM}{}{}{}

\cventry{2019}{Enterprise IT Transformation Advisor Level 4}{IBM}{}{}{}
\cventry{2019}{Banking Industry Foundations}{IBM}{}{}{}
\cventry{2019}{Telecommunications Industry Foundations}{IBM}{}{}{}
\cventry{2019}{IBM Mentor}{IBM}{}{}{}
\cventry{2019}{IBM Cloud Essentials}{IBM}{}{}{}
\cventry{2019}{Enterprise Design Thinking Practitioner}{IBM}{}{}{}

\cventry{2019}{Информационная безопасность, повышение квалификации (772406077178)}{УЦ Эшелон}{Москва}{}{}

\cventry{2016}{Управление распределенной командой}{Нетология}{Москва}{}{}
\cventry{2014}{Молекулярная биология и генетика}{Институт Биоинформатики}{Санкт-Петербург}{100\%}{}

\end{document}
